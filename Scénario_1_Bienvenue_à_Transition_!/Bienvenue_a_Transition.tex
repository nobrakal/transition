\documentclass[10pt,a4paper,twocolumn]{article}
\usepackage[utf8]{inputenc}
\usepackage{lmodern,textcomp}
\usepackage[french]{babel}
\usepackage[T1]{fontenc}
\usepackage[linkcolor=blue, colorlinks=true,]{hyperref}
\author{Projet Transition}
\title{Bienvenue à Transition !}
\begin{document}
\maketitle
\section{Introduction}
Ce scénario a été écrit dans l'idée de faire découvrir la ville de Transition à l'époque contemporaine. Il est "règles-free", c'est à dire que vous devriez pouvoir l'adapter à presque tous les systèmes de règles qui existent.
\subsection{Résumé}
Arnaud Émery, une des petites mains de Newcommen (un agent double auprès de Dicealy) fait des siennes. Par des vols de documents confidentiels, il réussit à se mettre la Digitale à dos. Cette dernière lance une procédure d'interception d'Émery dans le métro parisien et ce dernier, acculé, est obligé de faire appel à des inconnus. Les Dicealy, qui suivent les documents (qui ont une puce GPS), vont vite se rendre compte que ces bleus peuvent les aider dans leur guerre souterraine avec les Newcommen.\\
Par dessus, Newcommen décide d'éliminer Arnaud (il s'est fait prendre et n'est donc plus sûr, même s'il s'en est réchappé) et le fait tuer.\\
Ce dernier, en mourant décide de rapporter les documents à Dicealy et utilise à cette fin les joueurs. Dicealy verront eux une aubaine d'avoir des agents doubles, et chargeront les joueurs d'aller donner les documents eux-mêmes à Newcommen, le tout bardé de quelques senseurs.
\subsection{Les personnages}
Une bande d'adolescents français fera très bien l'affaire. En réalité, même n'importe quel groupe de personnes ayant une bonne raison de se retrouver dans le métro parisien en soirée peut tout à fait jouer ce scénario.


\section{À l'aide!}
Alors que les joueurs sont dans le métro parisien, dans la correspondance entre le métro 1 et le RER A gare de Lyon pour être précis, un soir de peu d'affluence, un homme cours vers eux. Ce  dernier, grand, plutôt mince et vêtu d'un long impair noir a l'air d'avoir peur. Dès qu'il est à porté de voix, il interpelle les joueurs et leur dit:
\begin{quote}
S'il vous plait, à l'aide! Je suis poursuivi, des types veulent ma mort! J'ai juste besoin de me cacher parmi vous quelques minutes. S'il vous plait !
\end{quote}
Si les joueurs sont récalcitrants, il montre quelques billets. S'ils demandent son identité, il dit s'appeler Arnaud.
\\
Dès que la chose est actée, le manteau se transforme (il se colle à Arnaud et change de couleur et de texture) en pull/jean, dont il met la capuche. Arnaud lance alors la discussion sur le dernier film en vogue.\\
Exactement une minute plus tard, 2 hommes en costume trois pièces arrivent d'un pas décidé (ce sont des gardes de la Digitale). Ils passent devant notre groupe sans ciller. Ils se retournent au bout de 10 mètres, leur senseurs ayant captés la présence d'Arnaud. Combat bref mais tendu, les joueurs doivent sentir la supériorité matérielle des deux partis (surtout concernant les armes, elles ne tirent pas à une cadence normale...). Arnaud prend alors la parole, tout en tendant une carte bancaire aux joueurs:
\begin{quote}
Désolé de vous avoir mis dans l'embarras, mais vous risquez d'avoir des problèmes maintenant. Prenez ça en gage de remerciement et pour votre silence. Maintenant, courrez!
\end{quote}
Des bruits de pas se font en effet entendre depuis un bout du couloir.

\section{Jackpot ?}
La carte bancaire est une visa gold, sans nom apparent. Le compte qui lui est rattaché est bloqué mais contient 1000 €. La deuxième dois qu'ils s'en servent, un message apparait sur la machine (ou sur le ticket): 
\begin{quote}
RDV comme la dernière fois. Urgent. Besoin de votre aide.
\end{quote}
C'est \hyperlink{arnaud}{Arnaud} qui a besoin d'aide. Il vient de se faire mettre à la rue par Newcommen. Nul doute que l'appât du gain risque de motiver nos joueurs. \\
Quand ils se rendent à l'endroit indiqué, ils aperçoivent 5 types en costume, dont un, sortant discrètement un couteau et poignardant un SDF. Ils jettent deux/trois regards et s'en vont comme si de rien était.
\\
Il s'agit bien sûr d'Arnaud, que les gardes de la Newcommen ont retrouvés avant nos joueurs. Il agonise.
\begin{quote}
Ils m'ont eu. Rendez-moi un dernier service... La carte et... *il montre sa bague, une tête de mort*. Dans l'arrière-boutique du magasin de tapis de la porte Montreuil... Venez de la part de Thomas... Newcommen.
\end{quote}
Il rend son dernier souffle.
Si nos joueurs sont avides, ils peuvent trouver quelques objets intéressants sur le corps du malheureux Arnaud.
Nos amis ont en tout cas tout intérêt à déguerpir au plus vite... Car les passants vont sans doute rappliquer et se faire un plaisir d'appeler la police.
\section{"Les tapis d'orient de Montreuil"}
Encouragez les joueurs à se poser des questions, même à avoir peur (ils croisent plus de gens en costumes dans la rue par exemple). \\
Le magasin de tapis ("Les tapis d'orient de Montreuil") ne paye pas de mine, et fais carrément miteux. Un maghrébin à la mine patibulaire tient la boutique. Il s'agit d'un agent local des Dicealy. Les personnages sont libres d'aller soit dans l'arrière boutique directement, soit de parler au tenancier. Dans tous les cas, ce dernier se montrera très soupçonneux (ça n'est pas du tout un bon commercial), et refusera l'accès à l'arrière boutique. S'ils mentionnent le nom de Newcommen, il ira discrètement derrière son comptoir et, croyant bien faire, va enclencher la procédure défense de la boutique.\\
La porte se referme toute seule, des barres métalliques la protègent. \hyperlink{abdel}{Abdel} (le gérant), sort un fusil mitrailleur:
\begin{quote}
Qu'est ce que vous foutez-là ? Vous êtes qui ?!
\end{quote}
On comprendra qu'il réponde d'une manière assez violente... \\
S'en suit une petite scène de combat, qui peut vite mal tourner, surtout si les joueurs ne sont pas préparés. Deux solutions sont possibles:
\begin{itemize}
	\item Les joueurs se font battre. Ils sont fait prisonnier et remis à représentant Dicealy.
	\item Ils font prisonnier ou tuent Abdel, et le représentant Dicealy arrive.
\end{itemize}
Ce dernier s'appelle Philippe (il ne mentionne pas son nom). Il emmène les joueurs dans un bar à côté pour écouter leur histoire.\\
Germe alors très vite dans son esprit un plan: Envoyer les joueurs chez Newcommen, disant qu'ils sont commissionnés par Arnaud pour leur remettre les documents. Bien sûr, les joueurs devront en apprendre le plus possibles, et seront grassement rémunérés pour ça. \\
Il passe deux/trois coups de fils et emmène nos joueurs dans un sous-sol d'un immeuble voisin. Il ne répond que très peu aux questions des joueurs, se référant à sa supérieure qui doit leur en dire plus.

\section{Briefing}

Ils se retrouvent devant une entrée de cave. Au moment où Philippe s'approche de la prote, deux hommes surgissent, échangent quelques mots avec ce dernier, et ouvrent la porte. Le sous-sol est un des quartiers généraux de la Digitale et donc est truffée de militaires et de matériel technologique (au mépris des règles de Transition). Philippe guide les joueurs assez profondément jusqu'à) une salle d'apparence anodine avec une table au centre. Une belle femme (grande, blonde), Alice Newcommen (assez éloignée de la branche de la famille régnante, mais elle a quand même un certain pouvoir) entre. C'est la supérieure de Philippe, et elle n'a pas l'air de rigoler.\\
Elle explique aux joueurs ce qui se passe: Des forces plus grandes qu'eux existent et se battent. Newcommen est "attaquée" par Dicealy, et les joueurs en tant qu'inconnus feront de très bonnes taupes. Il leur suffit d'aller à Transition et de se faire recruter chez Dicealy.
\\
C'est une mission à prendre ou... à prendre.

\section{Newcommen nous voici}
Ici les personnages vont se faire recruter chez les Newcommen, sur ordre des Dicealy afin des les infiltrer.

\section{Abandon}
Les Dicealy vont abandonner leurs nouveaux "amis" et les sacrifier pour leur guerre contre Newcommen. Comment vont réagir les joueurs ?
\newpage
\section*{Les personnages non-joueurs}
\hypertarget{arnaud}{\paragraph{Arnaud}} Agent double Newcommen/Dicealy\\ \\
\
\begin{tabular}{l|p{0.3\textwidth}}
	Âge & 30 ans\\
	Corpulence & Maigre \\
	Arme & Beretta 102 (19mm) \\
	Possessions & \begin{itemize}
	\item Une veste kméléon. Ils ne savent cependant pas comment la faire fonctionner.
	\item Des faux papiers au nom de Jospeh Priou.
	\item 20 € en liquide.
\end{itemize} \\
\end{tabular}
\hypertarget{abdel}{\paragraph{Abdel}} Gardien de la porte des tapis de Montreuil\\ \\
\
\begin{tabular}{l|p{0.3\textwidth}}
	Âge & 40 ans\\
	Corpulence & Maigre \\
	Arme & Uzi (9mm) \\
	Possessions & \begin{itemize}
	\item 50€ en liquide.
	\item Divers téléphones.
\end{itemize}	 \\
\end{tabular}

\end{document}
