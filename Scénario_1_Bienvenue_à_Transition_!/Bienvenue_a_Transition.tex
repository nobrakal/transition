\documentclass[10pt,a4paper,twocolumn]{article}
\usepackage[utf8]{inputenc}
\usepackage{lmodern,textcomp}
\usepackage[french]{babel}
\usepackage[T1]{fontenc}
\usepackage[linkcolor=blue, colorlinks=true,]{hyperref}
\usepackage{xcolor}
\author{Projet Transition}
\title{Bienvenue à Transition !}
\newenvironment{lAbstract}[1]{{[}\textcolor{red}{#1}{]}\itshape}{\\ \\}
\begin{document}

\maketitle
\begin{abstract}
Arnaud Émery, une des petites mains de Newcommen fait des siennes. Par des vols de documents confidentiels, il réussit à se mettre la Digitale à dos. Cette dernière lance une procédure d'interception d'Émery dans le métro parisien et ce dernier, acculé, est obligé de faire appel à des inconnus. 
Quelques temps après, Newcommen décide d'éliminer Arnaud (il s'est fait prendre et n'est donc plus sûr, même s'il s'en est réchappé) et le fait tuer.\\
Ce dernier, en mourant décide de rapporter les documents à Dicealy et utilise à cette fin les joueurs. Dicealy verront eux une aubaine d'avoir des agents doubles, et chargeront les joueurs d'aller donner les documents eux-mêmes à Newcommen, le tout bardé de quelques senseurs.
\end{abstract}
\section{Introduction}
Ce scénario a été écrit dans l'idée de faire découvrir la ville de Transition à l'époque contemporaine. Il est "règles-free", c'est à dire que vous devriez pouvoir l'adapter à presque tous les systèmes de règles qui existent. Les caractéristiques et possessions des personnages sont résumés en dernière page.

\subsection*{Les personnages}
Une bande d'adolescents français fera très bien l'affaire. En réalité, même n'importe quel groupe de personnes ayant une bonne raison de se retrouver dans le métro parisien en soirée peut tout à fait jouer ce scénario.

\section{À l'aide!}
\begin{lAbstract}{Combat}
Ici les personnages, vont se rendre compte que venir à l'aide du premier venu n'est pas la meilleure chose à faire
\end{lAbstract}
Alors que les joueurs sont dans le métro parisien, dans la correspondance entre le métro 1 et le RER A gare de Lyon pour être précis, un soir de peu d'affluence, un homme cours vers eux. Ce  dernier, grand, plutôt mince et vêtu d'un long impair noir a l'air d'avoir peur. Dès qu'il est à porté de voix, il interpelle les joueurs et leur dit:
\begin{quote}
S'il vous plait, à l'aide! Je suis poursuivi, des types veulent ma mort! J'ai juste besoin de me cacher parmi vous quelques minutes. S'il vous plait !
\end{quote}
Si les joueurs sont récalcitrants, il montre quelques billets. S'ils demandent son identité, il dit s'appeler Arnaud.
\\
Dès que la chose est actée, le manteau se transforme (il se colle à Arnaud et change de couleur et de texture) en pull/jean, dont il met la capuche. Arnaud lance alors la discussion sur le dernier film en vogue.\\
Exactement une minute plus tard, 2 hommes de Vigipirate arrivent d'un pas décidé (ce sont des gardes de la Digitale). Ils passent devant notre groupe sans ciller. Ils se retournent au bout de 10 mètres, leur senseurs ayant captés la présence d'Arnaud. Combat bref mais tendu, les joueurs doivent sentir la supériorité matérielle des deux partis (surtout concernant les armes, elles ne tirent pas à une cadence normale...). Arnaud un garde, mais le second en fuite et prend alors la parole, tout en tendant une carte bancaire aux joueurs:
\begin{quote}
Désolé de vous avoir mis dans l'embarras, mais vous risquez d'avoir des problèmes maintenant. Prenez ça en gage de remerciement et pour votre silence. Maintenant, courrez!
\end{quote}
Des bruits de pas se font en effet entendre depuis un bout du couloir. \\
La carte bancaire contient des informations confidentielles de la Digitale, et elle est camouflée dans une carte bancaire exprès pour, qu'en cas de besoin, le détenteur puisse la donner à un commun.

\section{Jackpot ?}
\begin{lAbstract}{Histoire}
Ici les personnages vont se rendre compte qu'avoir aidé Arnaud était la pire des idée.
\end{lAbstract}
La carte bancaire est une visa gold, sans nom apparent. Le compte qui lui est rattaché est bloqué mais contient 1000 €. La deuxième fois qu'ils s'en servent, un message apparait sur la machine (ou sur le ticket): 
\begin{quote}
RDV comme la dernière fois. Urgent. Besoin de votre aide.
\end{quote}
C'est \hyperlink{arnaud}{Arnaud} qui a besoin d'aide. Il vient de se faire mettre à la rue par Newcommen. Nul doute que l'appât du gain risque de motiver nos joueurs. \\
Quand ils se rendent à l'endroit indiqué, ils aperçoivent 5 types en costume, dont un, sortant discrètement un couteau et poignardant un SDF. Ils jettent deux/trois regards et s'en vont comme si de rien était.
\\
Il s'agit bien sûr d'Arnaud, que les gardes de la Newcommen ont retrouvés avant nos joueurs. Il agonise.
\begin{quote}
Ils m'ont eu. Rendez-moi un dernier service... La carte... Dans l'arrière-boutique du magasin de tapis de la porte Montreuil... Venez de la part de Thomas... Newcommen.
\end{quote}
Il rend son dernier souffle.
Si nos joueurs sont avides, ils peuvent trouver quelques objets intéressants sur le corps du malheureux Arnaud.
Nos amis ont en tout cas tout intérêt à déguerpir au plus vite... Car les passants vont sans doute rappliquer et se faire un plaisir d'appeler la police.

\section{"Les tapis d'orient de Montreuil"}
\begin{lAbstract}{Combat}
Ici les personnages, croyant bien faire, vont mettre un garde Dicealy en rogne.
\end{lAbstract}
Encouragez les joueurs à se poser des questions, même à avoir peur (ils croisent plus de gens en costumes ou se font contrôler par des militaires de Vigipirate dans la rue par exemple). \\
Le magasin de tapis ("Les tapis d'orient de Montreuil") ne paye pas de mine, et fait carrément miteux. Un maghrébin à la mine patibulaire tient la boutique. Il s'agit d'un agent local des Dicealy. Les personnages sont libres d'aller soit dans l'arrière boutique directement, soit de parler au vendeur. Dans tous les cas, ce dernier se montrera très soupçonneux (ce n'est pas du tout un bon commercial) et refusera l'accès à l'arrière boutique. S'ils mentionnent le nom de Newcommen, il ira discrètement derrière son comptoir et, croyant bien faire, va enclencher la procédure défense de la boutique.\\
La porte se referme toute seule, des barres métalliques la protègent. \hyperlink{abdel}{Abdel} (le gérant), sort un fusil mitrailleur:
\begin{quote}
Qu'est ce que vous foutez-là ? Vous êtes qui ?!
\end{quote}{]}
On comprendra qu'il réponde d'une manière assez violente... \\
S'en suit une petite scène de combat, qui peut vite mal tourner, surtout si les joueurs ne sont pas préparés. Deux solutions sont possibles:
\begin{itemize}
	\item Les joueurs se font battre. Ils sont fait prisonnier et remis à représentant Dicealy.
	\item Ils font prisonnier ou tuent Abdel, et le représentant Dicealy arrive.
\end{itemize}
Ce dernier s'appelle Philippe (il ne mentionne pas son nom). Il est d'abord d'un ton plutôt, puis il emmène les joueurs dans un bar à côté pour écouter leur histoire.\\
Germe alors très vite dans son esprit un plan: Envoyer les joueurs chez Newcommen, disant qu'ils sont commissionnés par Arnaud pour leur remettre les documents. Bien sûr, les joueurs devront en apprendre le plus possibles, et seront grassement rémunérés pour ça. \\
Il passe deux/trois coups de fils et emmène nos joueurs dans un sous-sol d'un immeuble voisin (il leur fait comprendre qu'il serait mal avisé de ne pas le suivre). Il ne répond que très peu aux questions des joueurs, se référant à sa supérieure qui doit leur en dire plus.

\section{Briefing}
\begin{lAbstract}{Découverte}
Ici les personnages vont découvrir que le monde n'est pas que ce qu'ils connaissent et que des forces plus grandes qu'eux sont en jeu.
\end{lAbstract}
Ils se retrouvent devant une entrée de cave. Au moment où Philippe s'approche de la prote, deux hommes surgissent, échangent quelques mots avec ce dernier, et ouvrent la porte. Le sous-sol est un des quartiers généraux de la Digitale et donc est truffé de militaires et de matériel technologique (au mépris des règles de Transition). Philippe guide les joueurs assez profondément jusqu'à une salle d'apparence anodine avec une table au centre. Une belle femme (grande, blonde), Alice Dubois (une femme d'un petit rang chez Dicealy) entre (elle a un bras robotique). C'est la supérieure de Philippe, et elle n'a pas l'air de rigoler.\\
Elle explique aux joueurs ce qui se passe: Des forces plus grandes qu'eux existent et se battent. Dicealy est "attaquée" par Newcommen, et les joueurs en tant qu'inconnus feront de très bonnes taupes. Il leur suffit d'aller à Transition et d'approter les documents (elle parle de la carte bleue) à Newcommen. Cela fait partie de la mission de ne rien savoir; ils n'en seront que plus crédible auprès des recruteurs Dicealy. Bien sûr, ils seront guidés depuis le QG. Leur contact sera Philippe.\\
C'est une mission à prendre ou... à prendre. Si les personnages refusent, Alice n'hésite pas à les menacer directement.
\\
\\
Après ce rapide entretien, on leur prends leur vêtements pour quelques "ajouts" (puces GPS haute précision notamment ainsi que des appareils photos). Pendant ce cours intermède, ils sont conviés à faire quelques tests (un des joueurs apprendra notamment qu'il a un début de cancer à faire soigner assez rapidement, ce que le médecin présent fait très vite au moyen d'un résonateur).
\\
On leur rend leurs vêtements une petite demie-heure plus tard, ainsi qu'un plan touristique de Paris et les "documents", qui ont été convertis en virus informatique. Le passage de Beaujolais est soulignés, et on leur explique qu'il faudra aller dans les toilettes du restaurant de ce passage.
Philippe leur parle aussi du plan: Ils vont se rendre dans le restaurant, faire semblant d'être envoyé par Arnaud, et essayer de prendre un maximum de photos de tout. Une fois qu'ils auront donné les documents, ils devront partir au plus vite et rallier le parc-aux-fleurs. 

\section{Newcommen nous voici}
\begin{lAbstract}{Découverte}
Ici les personnages vont infiltrer les Newcommen sur ordre des Dicealy.
\end{lAbstract}
Direction donc le passage de Beaujolais, près du Palais-Royal. Le restaurant est d'un standing moyen, personne ne fera de problème pour entrer. C'est une porte sous contrôle de Newcommen. Philippe a prévenu les joueurs qu'il faudra mentionner le nom de Arnaud Émery afin qu'on les laisse passer. \\
Les gardes Newcommen sont d'abords décontenancés, surtout si les joueurs parlent des documents. Ils sont vite conduits à Transition puis dans le centre ville.
\\
Une fois dans le centre-ville (les personnages accompagnés des gardes Newcommen passent les contrôles sans problème), on le conduit dans un grand bâtiment. Il s'agit de l'immeuble des "affaires extérieures". Ils sont amené dans un petite salle au premier étage. Les gardes restent à l'extérieur, tandis qu'un officiel Newcommen (\hyperlink{henry}{Henry Newcommen}, un membre de la famille régnante mais à un degré assez éloigné).
Newcommen fustige le pauvre Arnaud (les responsables n'hésitent pas à l'insulter) d'avoir amené des inconnus à Transition. Mais bon, cela peut toujours servir, surtout si ces inconnus sont porteurs de documents confidentiels de Dicealy. La communication avec Philippe est coupée (les puces GPS et toutes traces de liens des joueurs avec Dicealy a été effacée).
Au milieu de cette conversation toute sympathique, Henry Newcommen va tester les documents devant nos joueurs. Et celui-ci va être plutôt surpris de voir l'ensemble du bâtiment simplement s'éteindre.

\section{Fuite}
\begin{lAbstract}{Fuite}
Il va falloir maintenant quitter le centre ville rapidement.
\end{lAbstract}
Les joueurs vont devoir agir vite. Henry est seul dans la pièce et seulement deux gardes gardent la porte (ils seront affaibli puisque le virus va atteindre leur armure connectée). Les joueurs pourront soit se déguiser, soit essayer de sortir dans la rue et de fuir. En tout cas le but est de partir le plus vite possible. Le quartier est néanmoins très grand et surtout, les joueurs ne savent pas où est le parc-aux-fleurs !

Improvisez! Jouez avec les joueurs dans cette partie. N'hésitez pas à mettre en scène des courses/poursuites et des embuscades.

Les personnages vont néanmoins arriver à rentrer au parc-aux-fleurs, où les gardes, mis au courant, les emmènent au QG de la digitale. Là-bas, Alice les remercie, leur annonce que leur compte a été crédité de 1000e, et que aux vues de leur compétences, ils seront ravis de re-travailler avec eux. En tout cas, elle leur remet des papiers temporaires Dicealy (il s'agit d'un smart-phone avec quelques ajouts), et rappelle les règles de l'Accord.

\section{Un retour à la vie normale ?}
\begin{lAbstract}{Réflexion}
Newcommen ne se laisse pas faire
\end{lAbstract}
Les joueurs vont se faire poursuivre dans le monde réel...

\newpage
\section*{Les personnages non-joueurs}
\hypertarget{arnaud}{\paragraph{Arnaud}} Agent double Newcommen/Dicealy\\ \\
\
\begin{tabular}{l|p{0.3\textwidth}}
	Âge & 30 ans\\
	Corpulence & Maigre \\
	Arme & Beretta 102 (19mm) \\
	Possessions & \begin{itemize}
	\item Une veste kméléon. Ils ne savent cependant pas comment la faire fonctionner.
	\item Des faux papiers au nom de Jospeh Priou.
	\item 20 € en liquide.
\end{itemize} \\
\end{tabular}
\hypertarget{abdel}{\paragraph{Abdel}} Gardien de la porte des tapis de Montreuil\\ \\
\
\begin{tabular}{l|p{0.3\textwidth}}
	Âge & 40 ans\\
	Corpulence & Maigre \\
	Arme & Uzi (9mm) \\
	Possessions & \begin{itemize}
	\item 50€ en liquide.
	\item Divers téléphones.
\end{itemize}	 \\
\end{tabular}

\hypertarget{henry}{\paragraph{Henry Newcommen}}Responsable Newcommen\\ \\
\
\begin{tabular}{l|p{0.3\textwidth}}
	Âge & 25 ans\\
	Corpulence & Normale \\
	Arme & Beretta 102 (19mm) \\
	Possessions & \begin{itemize}
	\item Pass pour le bâtiment des affaires extérieures Newcommen
	\item Talkie/Walkie
	\item Papiers au nom de Henry Newcommen
\end{itemize}	 \\
\end{tabular}

\end{document}
