\documentclass[10pt,a4paper,twocolumn]{article}
\usepackage[utf8]{inputenc}
\usepackage{lmodern,textcomp}
\usepackage[french]{babel}
\usepackage[T1]{fontenc}
\author{Projet Transition}
\title{Bienvenue à Transition !}
\begin{document}
\maketitle
\section{Introduction}
Ce scénario a été écrit dans l'idée de faire découvrir la ville de Transition à l'époque contemporaine. Il est "règles-free", c'est à dire que vous devriez pouvoir l'adapter à presque tous les systèmes de règles qui existent.
\subsection{Résumé}
Arnaud Émery, une des étoiles montantes de Newcommen fait des siennes. Par des vols de documents confidentiels, il réussit à se mettre la Digitale à dos. Cette dernière lance une procédure d'interception d'Émery dans le métro parisien et ce dernier, acculé, est obligé de faire appel à des inconnus. Il saisira bien vite que ces derniers peuvent lui être d'une grande aide.
\subsection{Les personnages}
Une bande d'adolescents français fera très bien l'affaire. En réalité, même n'importe quel groupe de personnes ayant une bonne raison de se retrouver dans le métro parisien en soirée peut tout à fait jouer ce scénario.


\section{À l'aide!}
Alors que les joueurs sont dans le métro parisien, dans la correspondance entre le métro 1 et le rer A gare de Lyon pour être précis, un soir de peu d'affluence, un homme cours vers eux. Ce  dernier, grand, plutôt mince et vêtu d'un long impair noir a l'air d'avoir peur. Dès qu'il est à porté de voix, il interpelle les joueurs et leur dit:
\begin{quote}
S'il vous plait, à l'aide! Je suis poursuivi, des types veulent ma mort! J'ai juste besoin de me cacher parmi vous quelques minutes. S'il vous plait !
\end{quote}
Si les joueurs sont récalcitrants, il montre quelques billets. S'ils demandent son identité, il dit s'appeler Arnaud.
\\
Dès que la chose est actée, le manteau se transforme (il se colle à Arnaud et change de couleur et de texture) en pull/jean, dont il met la capuche. Arnaud lance alors la discussion sur le dernier film en vogue.\\
Exactement une minute plus tard, 2 hommes en costume trois pièces arrivent d'un pas décidé (ce sont des gardes de la Digitale). Ils passent devant notre groupe sans ciller. Ils se retournent au bout de 10 mètres, leur senseurs ayant captés la présence d'Arnaud. Combat bref mais tendu, les joueurs doivent sentir la supériorité matérielle des deux partis. Arnaud prend alors la parole, tout en tendant une carte bancaire aux joueurs:
\begin{quote}
Désolé de vous avoir mis dans l'embarras, mais vous risquez d'avoir des problèmes maintenant. Prenez ça en gage de remerciement et pour votre silence. Maintenant, courrez!
\end{quote}
Des bruits de pas se font en effet entendre depuis un bout du couloir.
\section{Jackpot ?}
La carte bancaire est une visa gold, sans nom apparent. Le compte qui lui est rattaché est bloqué mais contient 1000 €. La deuxième dois qu'ils s'en servent, un message apparait sur la machine (ou sur le ticket): 
\begin{quote}
RDV comme la dernière fois. Urgent.
\end{quote}
C'est Arnaud qui a besoin d'aide, et nul doute que l'appât du gain risque de motiver nos joueurs. \\
Quand ils se rendent à l'endroit indiqué, ils aperçoivent 5 types en costume, dont un, sortant discrètement un couteau et poignardant un SDF. Ils jettent deux/trois regards et s'en vont comme si de rien était.
\\
Il s'agit bien sûr d'Arnaud, que les gardes de la Digitale ont retrouvés avant nos joueurs. Il agonise
\begin{quote}
Ils m'ont eu. Rendez-moi un dernier service... La carte et... *il montre sa bague, une tête de mort*. Dans l'arrière-boutique du magasin de tapis de la porte Saint-Cloud... Venez de la part de Thomas... Newcommen.
\end{quote}
Il rend son dernier souffle.
\paragraph*{Possessions}
Si nos joueurs sont avides, ils peuvent trouver sur le corps du malheureux Arnaud:
\begin{itemize}
	\item
\end{itemize}
\end{document}
