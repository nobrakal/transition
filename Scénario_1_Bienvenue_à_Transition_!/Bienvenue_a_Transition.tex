\documentclass[10pt,a4paper,twocolumn]{article}
\usepackage[utf8]{inputenc}
\usepackage{lmodern,textcomp}
\usepackage[french]{babel}
\usepackage[T1]{fontenc}
\author{Projet Transition}
\title{Bienvenue à Transition !}
\begin{document}
\maketitle
\section{Introduction}
Ce scénario a été écrit dans l'idée de faire découvrir la ville de Transition à l'époque contemporaine. Il est "règles-free", c'est à dire que vous devriez pouvoir l'adapter à presque tous les systèmes de règles qui existent.
\subsection{Résumé}
Arnaud Émery, une des étoiles montantes de Newcommen fait des siennes. Par des vols de documents confidentiels, il réussit à se mettre la Digitale à dos. Cette dernière lance une procédure d'interception d'Émery dans le métro parisien et ce dernier, acculé, est obligé de faire appel à des inconnus. Il saisira bien vite que ces derniers peuvent lui être d'une grande aide.
\subsection{Les personnages}
Une bande d'adolescents français fera très bien l'affaire. En réalité, même n'importe quel groupe de personnes ayant une bonne raison de se retrouver dans le métro parisien en soirée peut tout à fait jouer ce scénario.


\section{À l'aide!}
\subsection{Une soirée banale}
Alors que les joueurs sont dans le métro parisien, un soir de peu d'affluence, un homme cours vers eux. Ce dernier, grand, plutôt mince et vêtu d'un long impair noir a l'air d'avoir peur. Dès qu'il est à porté de voix, il interpelle les joueurs et leur dit:
\begin{quote}
S'il vous plait, à l'aide! Je suis poursuivi, des types veulent ma mort! J'ai juste besoin de me cacher parmi vous quelques minutes. S'il vous plait !
\end{quote}
Si les joueurs sont récalcitrants, il montre quelques billets. S'ils demandent son identité, il dit s'appeler Arnaud.
\\
Dès que la chose est actée, le manteau se transforme (il se colle à Arnaud et change de couleur et de texture) en pull/jean, dont il met la capuche. Arnaud lance alors la discussion sur le dernier film en vogue.\\
Exactement une minute plus tard, 2 hommes en costume trois pièces arrivent d'un pas décidé (ce sont des gardes de la Digitale). Ils passent devant notre groupe sans ciller. Ils se retournent au bout de 10 mètre, leur senseurs ayant captés la présence d'Arnaud. Cependant, ce dernier a déjà sortit son arme et tire deux coups discret, les deux gardes sont tués sur le champs. Arnaud prend alors la parole, tout en tendant une carte bancaire aux joueurs.
\begin{quote}
Désolé de vous avoir mis dans l'embarras, mais vous risquez d'avoir des problèmes maintenant. Prenez ça en gage de remerciement et pour votre silence. Maintenant, courrez!
\end{quote}
Des bruits de pas se font en effet entendre depuis un bout du couloir.
\subsection{Jackpot ?}
La carte bancaire est une visa gold, sans nom apparent. Le compte qui lui est rattaché est bloqué mais contient 1000 €. Dès qu'ils s'en servent, un message apparait sur la machine (ou sur le ticket): 
\begin{quote}
RDV comme la dernière fois. Urgent.
\end{quote}
\end{document}
