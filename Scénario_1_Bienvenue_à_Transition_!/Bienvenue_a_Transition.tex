\documentclass[10pt,a4paper,twocolumn]{article}
\usepackage[utf8]{inputenc}
\usepackage[french]{babel}
\usepackage[T1]{fontenc}
\author{Projet Transition}
\title{Bienvenue à Transition !}
\begin{document}
\maketitle
\section{Introduction}
Ce scénario a été écrit dans l'idée de faire découvrir la ville de Transition à l'époque contemporaine. Il est "règles-free", c'est à dire que vous devriez pouvoir l'adapter à presque tous les systèmes de règles qui existent.
\subsection{Résumé}
Joseph Émery, une des étoiles montantes de Newcommen fait des siennes. Par des vols de documents confidentiels, il réussit à se mettre la Digitale à dos et bien sûr à se faire prendre en plein monde commun, devant des témoins en plus. Mais peut-être que tout ça a été voulu.
\subsection{Les personnages}
Une bande d'adolescents français fera très bien l'affaire. En réalité, même n'importe quel groupe de personnes ayant une bonne raison de se retrouver dans un centre commercial à la Défense peut tout à fait jouer ce scénario.


\section{Attentat à la Défense}
Alors que les personnages font leurs emplettes, un jour de peu d'affluence, quelqu'un en costume rentre dans la boutique et ferme le rideau de fer. Il appuie ensuite discrètement sur une partie de son manteau et celui-ci semble se durcir.\\
Il aperçoit les joueurs, jure, et leur demande de se cacher \emph{rapidement}. Quelques secondes après, des rafales se font entendre et cinq hommes lourdement armés entrent dans le magasins et tirent sur le premier.\\
Alors que tout semble calmé et que les 5 hommes s'approchent du cadavre, une grosse explosion se fait entendre, et après que la poussière se soit dissipée, seul le premier est debout et s'avance vers les joueurs:
\begin{quote}
Excusez-moi pour ça. Je devrai vous tuer là maintenant selon les lois de mon... pays, mais vu que je suis en froid avec ce dernier, je crois que je vais vous laisser la vie sauve.
\end{quote}
Il pianote sur l'écran d'un téléphone portable %Il change l'identité des tués et appelle un régmient spécial de la police.
\begin{quote}
Voilà qui devrait régler l'affaire. Sur ce, faites comme-ci vous n'aviez rien vu, et croyez moi, vous feriez bien.
\end{quote}
Il s'éclipse alors qu'une sirène de police se fait entendre. En quelques minutes, quoi que fassent les personnages, le 7ème corps du GIGN intervient. À la surprise des joueurs, ces derniers sont raccompagnés par des gendarmes sûr-armés, plutôt peu loquaces, à une sortie discrète du centre commercial. S'ils laissent trainer leurs regards sur la scène du crime, ils voient que les gendarmes, à une vitesse incroyables, changent les habits des morts et déposent même de faux papiers sur eux.
\\
S'ils vérifient, le 7ème corps du GIGN n'existe pas.

\section{Ça n'est pas normal}
\end{document}
