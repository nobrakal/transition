\documentclass[10pt,a4paper,twocolumn]{article}
\usepackage[utf8]{inputenc}
\usepackage[french]{babel}
\usepackage[T1]{fontenc}
\author{Projet Transition}
\title{Bienvenue à Transition !}
\begin{document}
\maketitle
\section{Introduction}
Ce scénario a été écrit dans l'idée de faire découvrir la ville de Transition à l'époque contemporaine. Il est "règles-free", c'est à dire que vous devriez pouvoir l'adapter à presque tous les systèmes de règles qui existent.
\subsection{Résumé}
Arnaud Émery, une des étoiles montantes de Newcommen fait des siennes. Par des vols de documents confidentiels, il réussit à se mettre la Digitale à dos. Cette dernière lance une procédure d'interception d'Émery dans le métro parisien et ce dernier, acculé, est obligé de faire appel à des inconnus.
\subsection{Les personnages}
Une bande d'adolescents français fera très bien l'affaire. En réalité, même n'importe quel groupe de personnes ayant une bonne raison de se retrouver dans le métro parisien en soirée peut tout à fait jouer ce scénario.


\section{À l'aide!}
Alors que les joueurs sont dans le métro parisien, un soir de peu d'affluence, un homme cours vers eux. Ce dernier, grand, plutôt mince et vêtu d'un long impair noir a l'air d'avoir peur. Dès qu'il est à porté de voix, il interpelle les joueurs et leur dit:
\begin{quote}
S'il vous plait, à l'aide! Des types veulent ma mort! J'ai juste besoin de me cacher parmi vous, quelques minutes. S'il vous plait !
\end{quote}
Si les joueurs sont récalcitrants, il montre quelques billets. S'ils demandent son identité, il dit s'appeler Arnaud.
\\
Dès que la chose est actée, le manteau se transforme (il se colle à Arnaud et change de couleur et de texture) en pull/jean, dont il met la capuche. Arnaud lance alors la discussion sur le dernier film en vogue.\\
Exactement une minute plus tard, 5 hommes en long costume trois pièces arrivent d'un pas décidé. 
\end{document}
