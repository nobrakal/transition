\documentclass{book}
\usepackage[utf8]{inputenc}
\usepackage[T1]{fontenc}
\usepackage[french]{babel}
\usepackage[linkcolor=blue, colorlinks=true,]{hyperref}
\usepackage{xcolor}
\usepackage{graphicx}
\usepackage{fancyhdr}
\graphicspath{{img/}}
\usepackage{epigraph}
\pagestyle{fancy}
\fancyhead[C]{\rightmark}
\fancyhead[L]{}
\fancyhead[R]{}
\fancyfoot[LO]{\hyperlink {participation} {Envie de participer?}}
\fancyfoot[RE]{\hyperlink {participation} {Envie de participer?}}

\makeatletter
\let\insertdate\@date

% Transition est sous licence CC-BY-SA, présente dans le dossier d'origine, merci de la respecter!
%
% Page mère regroupant les autres pour une meilleure coordination des participants (communiquez uniquement le fichier modifié)
% Pour tout nouveau participant, il est conseillé d'aller voir un tutoriel LaTex (celui de Wikibooks par exemple).

\title{Transition \\ Devel}
\author{Projet Transition}
\date{\oldstylenums{\insertdate}}
\begin{document}
\maketitle
\setcounter{tocdepth}{1} %Génération du Sommaire.
\renewcommand{\contentsname}{Sommaire} 
\tableofcontents

%%%%%%%%%
\newpage
\section*{Introduction}
Les grandes histoires sont monnaies courantes sur Terre. Les épopées, les aventures, les vies extraordinaires ne sont pas si rares. Que ce soit la  vie de César, de Napoléon ou d'Hitler, la découverte de la pénicilline ou de l’électricité, tout cela est dû à un nombre incalculable de coïncidences. Mais pourquoi donc? Pourquoi un tel hasard, qui semblent impensable, peut-il exister? Ne serais-ce réellement que de simples coïncidences?
Évidemment, et tout esprit logique en conviendra, non. Autre chose est derrière tout ceci: le Démiurge.
\\
\\
Le Démiurge est une entité créatrice qui a formé le monde. Elle l'a d'abord façonné, créant montagnes, vallées, mers et océans. Ce n'est qu'une fois avoir pris le temps de contempler son œuvre qu'elle y a injecté la vie. Elle l'a regardé évoluer, grandir, avancer, et, non content d'avoir déjà fait tout cela, l'a aidé.
\\
C'est ici que l'on retrouve le fil de nos exploits humains si extraordinaires: le Démiurge y a participé. Car, c'est la seule caractéristique connue de cet être: il aime les histoires et les grandes aventures. C'est pourquoi il aide les hommes à en écrire, qu'elles soient bonnes ou mauvaises.
Il consigne d'ailleurs toutes ces épopées dans sa grande bibliothèque, à Transition.
\\
\\
Transition, c'est une ville un peut particulière, en dehors du monde connue, aux milieux des océans, résidence du Démiurge et de sa bibliothèque. N'importe qui peut y accéder, s'il connait un des portes qui y mène. Il y en a un peu partout, mais surtout dans les grandes villes et toutes mènent au pied de la ville, dans la (bien nommée) grande salle des portes.
\\
\\
Rares sont ceux ayant trouvé l'accès. Les plus connus sont J.R.R. Tolkien, J.K. Rowling, ou encore Michael-Ange, De Vinci et Bach en leurs temps. C'est là-bas, et en ayant parcourus les livres de la grande bibliothèque, qu'ils ont acquis leurs connaissances. Mais ce n'est pas tout. Transition est une ville, et ses habitants sont plus étranges les uns que les autres...

\subsection*{Note au lecteur}
Ce livre est destiné aux maitres de jeu et autres compteurs. Si vous êtes juste un joueur curieux, ne lisez pas la suite. Elle contient des informations qui peuvent nuire à votre immersion.

\chapter{La ville}
\epigraph{Ce n'est pas dans je ne sais quelle retraite que nous nous découvrirons: c'est sur la route, dans la ville, au milieu de la foule, chose parmi les choses, homme parmi les hommes.}{\textit{Jean-Paul Sartre}}
\section{Le temps qui passe}
Le temps passe de manière normale à Transition. Cependant, ses effets ne se font pas ressentir; pour dire les choses simplement, personne ne vieillit à Transition. Des gens de tout age et de toute époque se côtoient.\\
Peu de gens qui ont passé beaucoup de temps dans la ville se risquent à en sortir. Que ce soit par peur d'une inadéquation avec le monde moderne ou par superstition...

\section{Un peu de géographie}
Transition est située au milieu d'un grand océan. Personne n'a jamais vraiment pu déterminer la position exacte de la ville sur le globe (si même cette dernière est sur Terre).\\
Quelques poissons vivent dans les eaux de l'océan sans fin, toutes terrestres; certains pensent à une introduction humaine, d'autres pensent que c'est parce que Transition est située dans un coin d'océan qui n'a pas encore été découvert.\\
Au niveau de la météorologie, les temps est semblable à n'importe quelle ville côtière, avec même des tempêtes de temps en temps.

\subsection{Les fuyards}
Certains tentent de partir de Transition par l'océan sans fin. Des embarcations, grosses ou petites, voir même des avions ont déjà décollés de transition. Personne n'est jamais revenu des ces expéditions. Les justifications sont nombreuses. Les plus croyants disent que le démiurge ne veut pas que l'on découvre ce qui se cache derrière ses créations, les plus pragmatiques que partir dans un océan sans fin depuis une cité telle que Transition frôle la déficience mentale.
\\
\\
Cependant, cet océan à trouvé une utilité pour les grandes factions désirant se débarrasser de personnes indésirables. Elles les mettent tout simplement sur une barque, qui est envoyée dans l'océan.

\section{La salle des portes}

C'est l'unique entrée de Transition. Il s'agit d'une salle circulaire (il s'agit d'un tore) d'un diamètre jamais évalué avec précision. Les murs sont couverts de différentes portes, en différents bois et de différentes formes. Des échelles sont même présente pour accéder aux portes en hauteur.\\
Sur la paroi extérieure du tore, des ouvertures régulière mènent vers les différents quartiers de Transition.

\subsection{Vous avez dit unique ?}
Pardon, par unique, on entend communément les seules entrées accessibles sans trop de problèmes. Il paraitrait que le Démiurge à sa propre porte, ainsi que quelques riches commerçants de la ville...

\subsection{Les portes}
Les portes en elles-mêmes sont de tout style (allant du drap pendant à une tringle à une robuste port en chêne). Cependant, dès que l'on passe l'encadrement (en fait, même dès que l'une partie du corps), on est projeté vers la destination de la porte.
\\
\\
Il est à noter qu'une fois la porte détruite, sa jumelle va pointer vers une nouvelle porte, choisie au hasard. Cela implique notamment que l'on ne peut se fier à l'aspect d'une porte pour savoir où elle transporte et, chose plus embêtante, si une porte est détruite dans la salle des portes, il est impossible aux voyageurs l'ayant emprunté de faire demi-tour. C'est la raison de la présence des gardes dans cette salle d'une grande importance.

\subsubsection{De la confection des portes}
Quelques grands scientifiques (le très vénérable Banisa'd ou encore le plus moderne Jean Lebrun) ont étudiés les caractéristiques des portes. Ils ont notamment notés que les portes doivent de confections humaines (ainsi, une entrée naturelle ne peut jamais se transformer en porte). De plus, il semblerait que la taille de cette porte importe peu: Lebrun cite une \guillemotleft porte de maison de poupées mannequins \guillemotright \ ayant transporté un enfant dans la salle des portes.

\subsection{Les gardes}
Des gardes de chaque factions sont présents et surveillent les entrées, les sorties et que personne ne porte atteinte aux si précieuses portes.

\subsection{"Bienvenue à Transition"}
Les nouveaux arrivants sont très vite détectés par les gardes de chaque factions, qui viennent voir les nouveaux arrivants et prononcent la phrase rituelle "Bienvenue à Transition", avertissant tous ceux autour qu'ils s'agit de \hyperlink{neophytes}{néophytes}.
%TODO déroulement, passage de rites ?

\section{La grande bibliothèque}
\subsection{Description}
Elle est située au sommet de Transition, pile au dessus de la salle des portes. Il s'agit d'un immense bâtiment au nombre de fenêtres quasi infini.\\
Le nombre de pièce n'a jamais été évalué proprement, et certains habitués se sont même aperçut que des pièces apparaissaient et disparaissent de temps en temps.
\subsection{Les savants}
Les savants sont les gardes de la bibliothèques. Souvent des vieillards (ils sont ici depuis très longtemps, mais leur corps n'en pas le reflet de leur âge, n'oubliez pas qu'à Transition le temps ne s'écoule pas de la même manière), ils ne vaut cependant mieux pas s'y mesurer. En effet, ces personnes ont fait le serment de protéger la bibliothèque ainsi que son contenu (notamment, aucun livre n'est autorisé à quitter l'endroit). Ils sont rompus à l'art des combats.
\\
Les savants ne font pas de recherches pour les lecteurs. Bien que connaissant les moindres recoins de la bibliothèque, ils estiment que les livres doivent trouver leur lecteurs aussi bien que l'inverse.
\subsubsection{Rejoindre les savants}
Les savants jouissent d'un très grand respect. Cependant personne ne sait comment rejoindre leur rang, et si même ils recrutent. En effet, personne n'a jamais vu un savant mort. Certains racontent cependant qu'un manuel d'instruction est caché dans la bibliothèque...
\subsection{Contenu}
Les pièces sont encombrés de milliards de lires, de différentes sortes, langues et sujet. Ils sont a peu près classé par thématiques grâce au travail incessant des savants.\\
Ils essayent d'ailleurs de renseigner les lecteurs du mieux qu'ils peuvent, mais ne font de recherches pour eux.


\chapter{Les factions majeures}
\section{Des factions ?}
On appelle factions majeures les 7 grandes organisations qui gouvernent Transition.
\hypertarget{accord}{\section{L'Accord}}
De mémoire d'homme (et elle peut remonter loin pour les habitants de la ville), cet accord existe depuis toujours à Transition
\begin{quote}
	Les grandes factions de Transition actent:
	\begin{itemize}
		\item Les hommes du monde commun doivent subvenir eux-mêmes à leur besoins. Afin de protéger les habitants de Transition, la communication entre nos deux mondes doit être réduite à son minimum.
		\item Par conséquent, interdiction est faite à tous les habitants de Transition de communiquer quelques informations que ce soit aux hommes communs, sous peine de mort.
	\end{itemize}
\end{quote}
Cet accord est \emph{la pulpart du temps} respecté à la lettre, car lorsque les factions promettent la mort, elles se font un plaisir de la donner avec une lenteur qui n'a d'égale que la créativité du Démiurge.
\\
Cependant, certains hommes et certaines femmes ont été trop tentées par l'appât du gain ou de la renommée. C'est ainsi que le monde commun a connut la roue ou la machine à vapeur. 

\section{Newcommen}
\epigraph{Le savoir est la plus grande des richesses.}{\textit{Thomas Newcommen}}
Cette faction tire son nom de la famille Newcommen, qui dirige la faction depuis les années 1500. Son membre le plus connus dans le monde commun est Thomas Newcommen "inventeur" de la machine à vapeur. Il est aussi connu à Transition pour avoir briser l'Accord.

\subsection{Le centre ville}
Les Newcommen dirigent d'une poigne de faire le quartier dit du centre ville. C'est là bas que se trouvent une bonne partie des commerces de la ville.

\subsection{Les libraires}
Si vous avez besoin d'une information, ne cherchez pas plus loin, allez voir un libraire Newcommen. Ces gens là connaissent une multitude de choses, et on à peu près un livre sur chaque sujet. Il va falloir néanmoins être prêt à en payer le prix...

\subsection{La garde Newcommen}
On peut reconnaitre la garde Newommen par leurs complet veste/pantalon et chemise à col Mao d'un gris profond.

\section{Dicealy}
\epigraph{Vaincre les ténèbres par la science.}{\textit{Dicton Dicelay}}
La faction Dicelay \emph{vénère} (il est important d'insister sur ce mot) la science. La science est le remède à tous les maux, mais elle doit être utilisée avec parcimonie, due à son statut divin.

\subsection{Le parc-aux-fleurs}
Les Dicealy règnent en maitre sur la parc-aux-fleurs, un quartier situé en hauteur de Transition, réputé pour son calme. Ça n'est bien sûr qu'en surface.

\subsection{La garde Dicealy}
Véritable police religieuse, la garde Dicealy, communément appelée la Digitale (pour faire référence à la fois à son caractère mortel et à son hyper-informatisation).
Ces traits correspondants aux effets recherchés, la garde s'est même dotée d'un logo représentant la fleur:
\begin{figure}
    \centering
    \def\svgwidth{15em}
    \input{img/digitalis.pdf_tex}
\end{figure}

\hypertarget{neophytes}{\section{Les néophytes}}
C'est à cette faction que les nouveaux arrivants sont attribués. Ils y sont jusqu'à avoir choisi une autre faction, ou à se confirmer chez les néophytes.

\subsection{Il professore}
Il s'agit du QG des néophyte; enfin, plutôt des QGs: chaque quartier de factions majeure abrite son Il Professore. C'est ce que l'on peut approcher le mieux par une chaine hôtelière. La plupart de temps, c'est un lieu plutôt convivial ou il fait bon vivre. 
\\
Les Il Professore sont très bien gardés, jours et nuits, par la faction des Néophytes. Il serait dangereux de les sous-estimer.

%\chapter{Les factions mineurs}
%TODO Exemple de faction plus petite, règle de création pour les mjs.


%%%%%%%%%%%
\newpage
\section{Participation au projet}
\subsection{Comment?}
\hypertarget{participation}{}
Transition vous a plu? 
Envie de partager vos scénarios, vos suggestions ou vos idées?
\newline
Contactez-nous via l'interface github: \href {https://github.com/nobrakal/transition} {https://github.com/nobrakal/transition}
\subsection{Les créateurs}
Voici la liste de ceux qui ont participé au projet Transition. Leur aide fut, est et sera toujours très précieuse au projet. Merci encore!  
\begin{itemize}
\item Alexandre ’Nobrakal’ Moine 
\end{itemize}
%\subsection{Remerciements}
%Voilà maintenant la liste de ceux qui ont contribué à plus petite mesure, et souvent sans le savoir. 
\subsection{Licence}
Transition est un projet libre de droit, publié sous la licence Creative Commons BY-SA. C'est à dire que quiconque a la possibilité d'utiliser ce document (ainsi que toute les autres parties du projet), de le redistribuer et de le modifier. La seule obligation est de redistribuer le contenu (modifié ou non) sous les mêmes conditions.
\end{document}
