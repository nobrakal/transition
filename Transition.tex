\documentclass{book}
\usepackage[utf8]{inputenc}
\usepackage[T1]{fontenc}
\usepackage[francais]{babel}
\usepackage[linkcolor=blue, colorlinks=true,]{hyperref}
\usepackage{xcolor}
\usepackage{graphicx}
\usepackage{fancyhdr}
\pagestyle{fancy}
\fancyhead[C]{\rightmark}
\fancyhead[L]{}
\fancyhead[R]{}
\fancyfoot[LO]{\hyperlink {participation} {Envie de participer?}}
\fancyfoot[RE]{\hyperlink {participation} {Envie de participer?}}
\newcommand{\cadre}[1]{\newline\newline\colorbox{lightgray}{\begin{minipage}{\linewidth}#1\end{minipage}}}

\makeatletter
\let\insertdate\@date

% Transition est sous licence CC-BY-SA, présente dans le dossier d'origine, merci de la respecter!
%
% Page mère regroupant les autres pour une meilleure coordination des participants (communiquez uniquement le fichier modifié)
% Pour tout nouveau participant, il est conseillé d'aller voir le tutoriel sur le LaTex du site du zero.

\title{Transition \\ Devel}
\author{Projet Transition}
\date{\oldstylenums{\insertdate}}
\begin{document}
\maketitle
\setcounter{tocdepth}{2} %Génération du Sommaire.
\renewcommand{\contentsname}{Sommaire} 
\tableofcontents

%%%%%%%%%
\newpage
\section*{Introduction}
Les grandes histoires sont monnaies courantes sur Terre. Les épopées, les aventures, les vies extraordinaires ne sont pas si rares. Que ce soit la  vie de César, de Napoléon ou d'Hitler, la découverte de la pénicilline ou de l’électricité, tout cela est dû à un nombre incalculable de coïncidences. Mais pourquoi donc? Pourquoi un tel hasard, qui semblent impensable, peut-il exister? Ne serais-ce réellement que de simples coïncidences?
Évidemment, et tout esprit logique en conviendra, non. Autre chose est derrière tout ceci: le Démiurge.
\\
\\
Le Démiurge est une entité créatrice qui a formé le monde. Elle l'a d'abord façonné, créant montagnes, vallées, mers et océans. Ce n'est qu'une fois avoir pris le temps de contempler son œuvre qu'elle y a injecté la vie. Elle l'a regardé évoluer, grandir, avancer, et, non content d'avoir déjà fait tout cela, l'a aidé.
\\
C'est ici que l'on retrouve le fil de nos exploits humains si extraordinaires: le Démiurge y a participé. Car, c'est la seule caractéristique connue de cet être: il aime les histoires et les grandes aventures. C'est pourquoi il aide les hommes à en écrire, qu'elles soient bonnes ou mauvaises.
Il consigne d'ailleurs toutes ces épopées dans sa grande bibliothèque, à Transition.
\\
\\
Transition, c'est une ville un peut particulière, en dehors du monde connue, aux milieux des océans, résidence du Démiurge et de sa bibliothèque. N'importe qui peut y accéder, s'il connait un des portes qui y mène. Il y en a un peu partout, mais surtout dans les grandes villes et toutes mènent au pied de la ville, dans la (bien nommée) grande salle des portes.
\\
\\
Rares sont ceux ayant trouvé l'accès. Les plus connus sont J.R.R. Tolkien, J.K. Rowling, ou encore Michael-Ange, De Vinci et Bach en leurs temps. C'est là-bas, et en ayant parcourus les livres de la grande bibliothèque, qu'ils ont acquis leurs connaissances. Mais ce n'est pas tout. Transition est une ville, et ses habitants sont plus étranges les uns que les autres...
\\
\\
En effet, le temps ne s'écoule pas de la même manière là-bas. Les gens ne vieillissent pas, personne ne prend une ride. Ce n'est qu'en franchissant un portail que l'on reprend les âges qui nous avaient épargnés. Il faut donc prendre garde à ne pas y rester trop longtemps... Ou à y rester pour toujours.

\chapter{La ville}
\section{La salle des portes}
C'est l'unique entrée de Transition. Il s'agit d'une salle circulaire (il s'agit d'un tore) d'un diamètre jamais évalué avec précision. Les murs sont couverts de différentes portes, en différents bois et de différentes formes. Des échelles sont même présente pour accéder aux portes en hauteur.\\
Sur la paroi extérieure du tore, des ouvertures régulière mènent vers les différents quartiers de Transition.
\subsection{Vous avez dit unique ?}
Pardon, par unique, on entend communément les seules entrées accessibles sans trop de problèmes. Il paraitrait que le Démiurge à sa propre porte, ainsi que quelques riches commerçants de la ville...

\subsection{Les gardes}
Des gardes de chaque factions sont présents et surveillent les entrées et les sorties.

\subsection{Mise sous surveillance}
Chaque arrivant est mis sous surveillance par les factions. Il est après soit recruté, soit exclus.
%TODO déroulement, passage de rites ?

\section{Les quartiers}
\subsection{Le port}
Ce quartier ne porte son nom que par sa proximité avec l'océan sans fin. 
%TODO quel océan ??
\subsection{Le centre ville}
%TODO sorte de vieille ville (cf. Naples)

\section{La grande bibliothèque}
\subsection{Description}
Elle est située au sommet de Transition, pile au dessus de la salle des portes. Il s'agit d'un immense bâtiment au nombre de fenêtres quasi infini.\\
Le nombre de pièce n'a jamais été évalué proprement, et certains habitués se sont même aperçut que des pièces apparaissaient et disparaissent de temps en temps.
\subsection{Les savants}
Les savants sont les gardes de la bibliothèques. Souvent des vieillards, ils ne vaut cependant mieux pas s'y mesurer. En effet, ces personnes ont fait le serment de protéger la bibliothèque ainsi que son contenu (notamment, aucun livre n'est autorisé à quitter l'endroit. Ils sont rompus à l'art des combats.
%TODO comment sont-ils formés ?

\chapter{Les factions}
% Sorte de grande famille. Un affiliation à l'une des factions est quasi-obligaotire.
\section{Les factions majeures}
%TODO Exemple de factions importantes
\section{Les factions mineurs}
%TODO Exemple de faction plus petite, règle de création pour les mjs.


%%%%%%%%%%%
\newpage
\section{Participation au projet}
\subsection{Comment?}
\hypertarget{participation}{}
Transition vous a plu? 
Envie de partager vos scénarios, vos suggestions ou vos idées?
\newline
Contactez-nous via l'interface github: \href {https://github.com/nobrakal/transition} {https://github.com/nobrakal/transition}
\subsection{Les créateurs}
Voici la liste de ceux qui ont participé au projet Transition. Leur aide fut, est et sera toujours très précieuse au projet. Merci encore!  
\begin{itemize}
\item Alexandre ’Nobrakal’ Moine 
\end{itemize}
%\subsection{Remerciements}
%Voilà maintenant la liste de ceux qui ont contribué à plus petite mesure, et souvent sans le savoir. 
\subsection{Licence}
Transition est un projet libre de droit, publié sous la licence Creative Commons BY-SA. C'est à dire que quiconque a la possibilité d'utiliser ce document (ainsi que toute les autres parties du projet), de le redistribuer et de le modifier. La seule obligation est de redistribuer le contenu (modifié ou non) sous les mêmes conditions.
\end{document}